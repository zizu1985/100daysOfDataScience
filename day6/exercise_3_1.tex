% Options for packages loaded elsewhere
\PassOptionsToPackage{unicode}{hyperref}
\PassOptionsToPackage{hyphens}{url}
%
\documentclass[
]{article}
\usepackage{lmodern}
\usepackage{amssymb,amsmath}
\usepackage{ifxetex,ifluatex}
\ifnum 0\ifxetex 1\fi\ifluatex 1\fi=0 % if pdftex
  \usepackage[T1]{fontenc}
  \usepackage[utf8]{inputenc}
  \usepackage{textcomp} % provide euro and other symbols
\else % if luatex or xetex
  \usepackage{unicode-math}
  \defaultfontfeatures{Scale=MatchLowercase}
  \defaultfontfeatures[\rmfamily]{Ligatures=TeX,Scale=1}
\fi
% Use upquote if available, for straight quotes in verbatim environments
\IfFileExists{upquote.sty}{\usepackage{upquote}}{}
\IfFileExists{microtype.sty}{% use microtype if available
  \usepackage[]{microtype}
  \UseMicrotypeSet[protrusion]{basicmath} % disable protrusion for tt fonts
}{}
\makeatletter
\@ifundefined{KOMAClassName}{% if non-KOMA class
  \IfFileExists{parskip.sty}{%
    \usepackage{parskip}
  }{% else
    \setlength{\parindent}{0pt}
    \setlength{\parskip}{6pt plus 2pt minus 1pt}}
}{% if KOMA class
  \KOMAoptions{parskip=half}}
\makeatother
\usepackage{xcolor}
\IfFileExists{xurl.sty}{\usepackage{xurl}}{} % add URL line breaks if available
\IfFileExists{bookmark.sty}{\usepackage{bookmark}}{\usepackage{hyperref}}
\hypersetup{
  pdftitle={R Notebook},
  hidelinks,
  pdfcreator={LaTeX via pandoc}}
\urlstyle{same} % disable monospaced font for URLs
\usepackage[margin=1in]{geometry}
\usepackage{color}
\usepackage{fancyvrb}
\newcommand{\VerbBar}{|}
\newcommand{\VERB}{\Verb[commandchars=\\\{\}]}
\DefineVerbatimEnvironment{Highlighting}{Verbatim}{commandchars=\\\{\}}
% Add ',fontsize=\small' for more characters per line
\usepackage{framed}
\definecolor{shadecolor}{RGB}{248,248,248}
\newenvironment{Shaded}{\begin{snugshade}}{\end{snugshade}}
\newcommand{\AlertTok}[1]{\textcolor[rgb]{0.94,0.16,0.16}{#1}}
\newcommand{\AnnotationTok}[1]{\textcolor[rgb]{0.56,0.35,0.01}{\textbf{\textit{#1}}}}
\newcommand{\AttributeTok}[1]{\textcolor[rgb]{0.77,0.63,0.00}{#1}}
\newcommand{\BaseNTok}[1]{\textcolor[rgb]{0.00,0.00,0.81}{#1}}
\newcommand{\BuiltInTok}[1]{#1}
\newcommand{\CharTok}[1]{\textcolor[rgb]{0.31,0.60,0.02}{#1}}
\newcommand{\CommentTok}[1]{\textcolor[rgb]{0.56,0.35,0.01}{\textit{#1}}}
\newcommand{\CommentVarTok}[1]{\textcolor[rgb]{0.56,0.35,0.01}{\textbf{\textit{#1}}}}
\newcommand{\ConstantTok}[1]{\textcolor[rgb]{0.00,0.00,0.00}{#1}}
\newcommand{\ControlFlowTok}[1]{\textcolor[rgb]{0.13,0.29,0.53}{\textbf{#1}}}
\newcommand{\DataTypeTok}[1]{\textcolor[rgb]{0.13,0.29,0.53}{#1}}
\newcommand{\DecValTok}[1]{\textcolor[rgb]{0.00,0.00,0.81}{#1}}
\newcommand{\DocumentationTok}[1]{\textcolor[rgb]{0.56,0.35,0.01}{\textbf{\textit{#1}}}}
\newcommand{\ErrorTok}[1]{\textcolor[rgb]{0.64,0.00,0.00}{\textbf{#1}}}
\newcommand{\ExtensionTok}[1]{#1}
\newcommand{\FloatTok}[1]{\textcolor[rgb]{0.00,0.00,0.81}{#1}}
\newcommand{\FunctionTok}[1]{\textcolor[rgb]{0.00,0.00,0.00}{#1}}
\newcommand{\ImportTok}[1]{#1}
\newcommand{\InformationTok}[1]{\textcolor[rgb]{0.56,0.35,0.01}{\textbf{\textit{#1}}}}
\newcommand{\KeywordTok}[1]{\textcolor[rgb]{0.13,0.29,0.53}{\textbf{#1}}}
\newcommand{\NormalTok}[1]{#1}
\newcommand{\OperatorTok}[1]{\textcolor[rgb]{0.81,0.36,0.00}{\textbf{#1}}}
\newcommand{\OtherTok}[1]{\textcolor[rgb]{0.56,0.35,0.01}{#1}}
\newcommand{\PreprocessorTok}[1]{\textcolor[rgb]{0.56,0.35,0.01}{\textit{#1}}}
\newcommand{\RegionMarkerTok}[1]{#1}
\newcommand{\SpecialCharTok}[1]{\textcolor[rgb]{0.00,0.00,0.00}{#1}}
\newcommand{\SpecialStringTok}[1]{\textcolor[rgb]{0.31,0.60,0.02}{#1}}
\newcommand{\StringTok}[1]{\textcolor[rgb]{0.31,0.60,0.02}{#1}}
\newcommand{\VariableTok}[1]{\textcolor[rgb]{0.00,0.00,0.00}{#1}}
\newcommand{\VerbatimStringTok}[1]{\textcolor[rgb]{0.31,0.60,0.02}{#1}}
\newcommand{\WarningTok}[1]{\textcolor[rgb]{0.56,0.35,0.01}{\textbf{\textit{#1}}}}
\usepackage{graphicx,grffile}
\makeatletter
\def\maxwidth{\ifdim\Gin@nat@width>\linewidth\linewidth\else\Gin@nat@width\fi}
\def\maxheight{\ifdim\Gin@nat@height>\textheight\textheight\else\Gin@nat@height\fi}
\makeatother
% Scale images if necessary, so that they will not overflow the page
% margins by default, and it is still possible to overwrite the defaults
% using explicit options in \includegraphics[width, height, ...]{}
\setkeys{Gin}{width=\maxwidth,height=\maxheight,keepaspectratio}
% Set default figure placement to htbp
\makeatletter
\def\fps@figure{htbp}
\makeatother
\setlength{\emergencystretch}{3em} % prevent overfull lines
\providecommand{\tightlist}{%
  \setlength{\itemsep}{0pt}\setlength{\parskip}{0pt}}
\setcounter{secnumdepth}{-\maxdimen} % remove section numbering

\title{R Notebook}
\author{}
\date{\vspace{-2.5em}}

\begin{document}
\maketitle

Exercise 3-1 \#\#\#\#\#\#\#\#\#\#\#\#\#

We have 2 continous variable from hicker smarthphone. dist
-\textgreater{} Distance (in km) alti -\textgreater{} Altitude (in m)

\begin{enumerate}
\def\labelenumi{\alph{enumi})}
\tightlist
\item
  Calculate the arithmetic mean and median for both distance and
  altitude
\end{enumerate}

Mean, Median for Distance

\begin{Shaded}
\begin{Highlighting}[]
\NormalTok{dist<-}\KeywordTok{c}\NormalTok{(}\FloatTok{12.5}\NormalTok{,}\FloatTok{29.9}\NormalTok{,}\FloatTok{14.8}\NormalTok{,}\FloatTok{18.7}\NormalTok{,}\FloatTok{7.6}\NormalTok{,}\FloatTok{16.2}\NormalTok{,}\FloatTok{16.5}\NormalTok{,}\FloatTok{27.4}\NormalTok{,}\FloatTok{12.1}\NormalTok{,}\FloatTok{17.5}\NormalTok{)}
\KeywordTok{mean}\NormalTok{(dist)}
\end{Highlighting}
\end{Shaded}

\begin{verbatim}
## [1] 17.32
\end{verbatim}

\begin{Shaded}
\begin{Highlighting}[]
\KeywordTok{median}\NormalTok{(dist)}
\end{Highlighting}
\end{Shaded}

\begin{verbatim}
## [1] 16.35
\end{verbatim}

Mean, Median for Altitude

\begin{Shaded}
\begin{Highlighting}[]
\NormalTok{alti<-}\KeywordTok{c}\NormalTok{(}\DecValTok{342}\NormalTok{,}\DecValTok{1245}\NormalTok{,}\DecValTok{502}\NormalTok{,}\DecValTok{555}\NormalTok{,}\DecValTok{398}\NormalTok{,}\DecValTok{670}\NormalTok{,}\DecValTok{796}\NormalTok{,}\DecValTok{912}\NormalTok{,}\DecValTok{238}\NormalTok{,}\DecValTok{466}\NormalTok{)}
\KeywordTok{mean}\NormalTok{(alti)}
\end{Highlighting}
\end{Shaded}

\begin{verbatim}
## [1] 612.4
\end{verbatim}

\begin{Shaded}
\begin{Highlighting}[]
\KeywordTok{median}\NormalTok{(alti)}
\end{Highlighting}
\end{Shaded}

\begin{verbatim}
## [1] 528.5
\end{verbatim}

\begin{enumerate}
\def\labelenumi{\alph{enumi})}
\setcounter{enumi}{1}
\tightlist
\item
  Determine the first and third quartiles for both distance and the
  altitude variables Discuss the shape of the distribution give the
  result of (a) and (b)
\end{enumerate}

Quartiles for distance

\begin{Shaded}
\begin{Highlighting}[]
\KeywordTok{quantile}\NormalTok{(dist,}\DataTypeTok{probs=}\KeywordTok{c}\NormalTok{(}\FloatTok{0.25}\NormalTok{,}\FloatTok{0.75}\NormalTok{),}\DataTypeTok{type=}\DecValTok{2}\NormalTok{)}
\end{Highlighting}
\end{Shaded}

\begin{verbatim}
##  25%  75% 
## 12.5 18.7
\end{verbatim}

To achieve correct quantile type 2 algorithm was used (default is 7).

Quartiles for distance

\begin{Shaded}
\begin{Highlighting}[]
\KeywordTok{quantile}\NormalTok{(alti,}\DataTypeTok{probs=}\KeywordTok{c}\NormalTok{(}\FloatTok{0.25}\NormalTok{,}\FloatTok{0.75}\NormalTok{),}\DataTypeTok{type=}\DecValTok{2}\NormalTok{)}
\end{Highlighting}
\end{Shaded}

\begin{verbatim}
## 25% 75% 
## 398 796
\end{verbatim}

What does it means? Distance. Mean is higher than median and mean in
higher range of 25\% and 75\% quantile. It means there were some trip(s)
with longer length than others.

Altitude. Mean is in the middle of 25\% and 75\% range. Mean is much
higher than average, so we have some trips with mich higher altitude
move then others.

Interquartile range for distance

\begin{Shaded}
\begin{Highlighting}[]
  \KeywordTok{quantile}\NormalTok{(dist,}\DataTypeTok{probs=}\KeywordTok{c}\NormalTok{(}\FloatTok{0.75}\NormalTok{),}\DataTypeTok{type=}\DecValTok{2}\NormalTok{) }\OperatorTok{-}\StringTok{ }\KeywordTok{quantile}\NormalTok{(dist,}\DataTypeTok{probs=}\KeywordTok{c}\NormalTok{(}\FloatTok{0.25}\NormalTok{),}\DataTypeTok{type=}\DecValTok{2}\NormalTok{)}
\end{Highlighting}
\end{Shaded}

\begin{verbatim}
## 75% 
## 6.2
\end{verbatim}

Interquartile range for distance

\begin{Shaded}
\begin{Highlighting}[]
  \KeywordTok{quantile}\NormalTok{(alti,}\DataTypeTok{probs=}\KeywordTok{c}\NormalTok{(}\FloatTok{0.75}\NormalTok{),}\DataTypeTok{type=}\DecValTok{2}\NormalTok{) }\OperatorTok{-}\StringTok{ }\KeywordTok{quantile}\NormalTok{(alti,}\DataTypeTok{probs=}\KeywordTok{c}\NormalTok{(}\FloatTok{0.25}\NormalTok{),}\DataTypeTok{type=}\DecValTok{2}\NormalTok{)}
\end{Highlighting}
\end{Shaded}

\begin{verbatim}
## 75% 
## 398
\end{verbatim}

Absolute median deviation

\begin{enumerate}
\def\labelenumi{\arabic{enumi}.}
\tightlist
\item
  For each observation do abs(x-mean). Output in vector t.
\item
  Sum all values in t.
\item
  Divide by len(t)
\end{enumerate}

Distance.

\begin{Shaded}
\begin{Highlighting}[]
\NormalTok{dist<-}\KeywordTok{c}\NormalTok{(}\FloatTok{12.5}\NormalTok{,}\FloatTok{29.9}\NormalTok{,}\FloatTok{14.8}\NormalTok{,}\FloatTok{18.7}\NormalTok{,}\FloatTok{7.6}\NormalTok{,}\FloatTok{16.2}\NormalTok{,}\FloatTok{16.5}\NormalTok{,}\FloatTok{27.4}\NormalTok{,}\FloatTok{12.1}\NormalTok{,}\FloatTok{17.5}\NormalTok{)}
\NormalTok{m<-}\KeywordTok{median}\NormalTok{(dist)}
\NormalTok{dist =}\StringTok{ }\NormalTok{dist }\OperatorTok{-}\StringTok{ }\NormalTok{m}
\NormalTok{dist =}\StringTok{ }\KeywordTok{abs}\NormalTok{(dist)}
\KeywordTok{sum}\NormalTok{(dist)}\OperatorTok{/}\KeywordTok{length}\NormalTok{(dist)}
\end{Highlighting}
\end{Shaded}

\begin{verbatim}
## [1] 4.68
\end{verbatim}

\end{document}
